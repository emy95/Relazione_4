\documentclass[12pt,]{article}

\usepackage[svgnames,table]{xcolor}
\usepackage{mathtools}
\usepackage[italian]{babel}
\usepackage{graphicx}
\usepackage[utf8]{inputenc}
\usepackage{float}
\usepackage{tabularx}
\usepackage{chngpage}
\usepackage[toc,page]{appendix}
\usepackage{gensymb}
\usepackage{subcaption}
\usepackage{tikz}
\usepackage{circuitikz}
\usepackage{array}
\usepackage{booktabs} 
\usepackage{colortbl} 
\usepackage{xcolor} 
\usepackage{xfrac}


\definecolor{burntorange}{cmyk}{0, 0.51,1,0}


\begin{document}
 \begin{titlepage}

\newcommand{\HRule}{\rule{\linewidth}{0.5mm}} % Defines a new command for the horizontal lines, change thickness here

\center % Center everything on the page
 
%----------------------------------------------------------------------------------------
%	HEADING SECTIONS
%----------------------------------------------------------------------------------------
\includegraphics[width = 50mm]{unitn.jpg}\\[0.5cm]
\textsc{\LARGE Università degli studi di Trento}\\[1.5cm] % Name of your university/college
\textsc{\Large Laboratorio di fisica II}\\[0.5cm] % Major heading such as course name
\textsc{\large Esperienza 2-3}\\[0.5cm] % Minor heading such as course title

%----------------------------------------------------------------------------------------
%	TITLE SECTION
%----------------------------------------------------------------------------------------

\HRule \\[0.2cm]
{ \huge \bfseries CIRCUITO RC E\\[0.1cm]FILTRI PASSA BASSO-ALTO}\\ % Title of your document
\HRule \\[1.5cm]
 
%----------------------------------------------------------------------------------------
%	AUTHOR SECTION
%----------------------------------------------------------------------------------------

\begin{minipage}{0.4\textwidth}
\begin{flushleft} \large
\emph{Autori:}\\
Canteri Marco\\Biasi Lorenzo\\Damiani Emily % Your name
\end{flushleft}
\end{minipage}
~
\begin{minipage}{0.4\textwidth}
\begin{flushright} \large
\emph{Professore:} \\
William J. Weber % Supervisor's Name
\end{flushright}
\end{minipage}\\[1.5cm]

% If you don't want a supervisor, uncomment the two lines below and remove the section above
%\Large \emph{Author:}\\
%John \textsc{Smith}\\[3cm] % Your name

%----------------------------------------------------------------------------------------
%	DATE SECTION
%----------------------------------------------------------------------------------------

{\large \today}\\[3cm] % Date, change the \today to a set date if you want to be precise

%----------------------------------------------------------------------------------------
%	LOGO SECTION
%----------------------------------------------------------------------------------------

%\includegraphics{Logo}\\[1cm] % Include a department/university logo - this will require the graphicx package
 
%----------------------------------------------------------------------------------------

\vfill % Fill the rest of the page with whitespace

\end{titlepage}

	\newpage
	\renewcommand{\abstractname}{Abstract}

	\begin{abstract}
In questa esperienza si è realizzato il cosiddetto Ponte di Wien con un condensatore dal valore incognito. Per farne una misura si è cercato di realizzare le condizioni per il bilanciamento del ponte. In un secondo momento, si è studiato la risposta del circuito a piccoli cambiamenti della capacità, attraverso lo oscilloscopio. In questo modo si è potuto quantificare con quale precisione il circuito risponde alla più piccola variazione di capacità apprezzabile. 

	\end{abstract}
  \vspace{10em}
\tableofcontents
  \newpage
  \section{Obiettivi}
  \begin{itemize}
  \item realizzare il ponte di Wien, utilizzando una capacità dal valore incognito, e ottenere le condizioni di bilanciamento del ponte. 
  \item confrontare il valore della capacità ottenuto teoricamente dalle condizioni di bilanciamento con il valore registrato dal multimetro digitale. 
  \item studiare come variano i valori di tensione in uscita dal circuito per  piccole variazioni della capacità. 
  \item quantificare con quanta precisione lo strumento reagisce a piccole variazioni di capacità. 
  \end{itemize}
  
\section{Materiali}
\begin{itemize}
\item breadbord
\item cavi coassiali 
\item 3 resistenze misurate con il multimetro digitale
\item 2 capacità
\item generatore di onda sinusoidale 
\item decade di capacità
\item decade di resistenza  
\item Oscilloscopio 
\end{itemize}
\section{Procedura di misura}
Nella prima parte dell'esperienza si è misurato il valore della capacità di un condensatore incognito. Per fare ciò si è costruito il circuito mostrato in figura, chiamato comunemente Ponte di Wien. Questo è stato costruito utilizzando il condensatore dal valore incognito, un altro condensatore e 3 resistenze, i cui valori sono mostrati in tabella: 

\begin{table}[H]
\centering
\begin{tabular}{c|c}
\toprule 
\midrule
$R_1$ &  $1007.2\pm 11.0720 \ohm$   \\
$R_2$ &  $997.80\pm 10.9780 \ohm$  \\
$R_x$ & $999.021 \pm 10.9902 \ohm$  \\
$C_x$ & $112 \cdot 10^{-9} F \pm $ \\
\bottomrule
\end{tabular}
\caption{Valori delle componenti del circuito misurate con il multimetro digitale}
\end{table}
Al circuito è stato collegato l'oscilloscopio attraverso i cavi coassiali, in modo tale da mostrare in uscita la differenza di potenziale tra i lati del ponte. Sistemato il tutto, si è proceduto al bilanciamento del ponte. Ciò equivale a cercare le condizioni di frequenza e di resistenze $R_r$ ( che può essere facilmente variata attraverso una decade) per cui la differenza di tensione tra i due lati del ponte è pari a 0. \\
Quindi, osservando lo schermo dell'oscilloscopio, e adottando un opportuno numero di medie sul segnale in uscita per ridurne il rumore, è possibile valutare per quali valori di frequenza e resistenza il segnale in uscita è pressoché nullo. Ottenuti questi valori, si è calcolato il valore della capacità prevista dalla teoria e verificata la compatibilità con il valore calcolato dal multimetro digitale. \newline \break
Una volta ottenuta la configurazione di bilanciamento, si è deciso di misurare la sensibilità del circuito a piccole variazione della capacità. Inizialmente si è posto in parallelo alla capacità $C$ una decade di capacità. Variando ogni volta la capacità in parallelo di $1 nF$ si sono prese 10 volte i valori della tensione in uscita letti dallo oscilloscopio, ripetendo la misura per 5 volte. Successivamente, sono stati presi altri dati con la decade in serie con il condensatore, seguendo lo stesso procedimento.\\
Attraverso i dati raccolti, si è poi potuto studiare in che modo il circuito rispondeva alle piccole variazioni di impedenza nella condizione di bilanciamento. 


\begin{figure}[h!]
\centering
  \begin{center}
\begin{circuitikz}
\draw (0,0)
node[ground]{}
to [sV= $ V_{in}$] (0,6)
to [short] (5,6)
to [short] (5,5.5)
to[short] (2.5,5.5)
to [vR= $ R_r $] (2.5,3)
to [short] (1.5, 3)
to[R=$ R_x $] (1.5, 0.75)
to[short](2.5, 0.75)
to[short](2.5,0.5)
;
\draw (2.5, 3)
to[short](3.5,3)
to[C=$ C_x$] (3.5, 0.75)
to[short] (2.5, 0.75)
;
\draw (5, 5.5)
to[short](7, 5.5)
to[R=$ R_1 $](7, 3)
to[R= $ R_2 $] (7, 2)
to[C= $ C_2 $] (7, 0.75)
to[short](7, 0.5)
to[short] (2.5, 0.5)
;
\draw (4.75, 0.5)
to[short] (4.75, 0)
node[ground]{}
;
\draw (2.5, 3.5)
to[short, -o](4, 3.5)node[right = 0.5em] {$V_{out}$}
;
\draw (7, 3.5)
to[short, -o](5.5, 3.5)
;
\end{circuitikz}

\caption{Ponte di Wien}
\label{fig:circuito}
\end{center}
\end{figure}




\section{Analisi dei dati}

\subsection{Bilanciamento del ponte e misura della capacità}
 Nella prima parte dell'esperienza si è eseguita la misura della capacità del condensatore sfruttando il bilanciamento del ponte di Wien. E' facile verificare che le condizioni per ottenere una differenza di potenziale tra i due bracci uguale a 0 sono 2: 
 \begin{equation}
 \frac{R_2}{R_x} + \frac{C_x}{C_2} = \frac{R_1}{R_r}
 \end{equation}
 \begin{equation}
 (2 \pi  f)^2 R_2  C_2  R_x  C_x = 1 
 \end{equation}
da cui si può esplicitare direttamente l'equazione per trovare il valore della capacità: 
\begin{equation}
C_x = \sqrt{\frac{1}{(2 \pi f)^2 R_2 R_x} \biggl(\frac{R_1}{R_r} - \frac{R_2}{R_x} \biggr)}
\end{equation}
Ora, dati due valori sperimentali ben precisi di frequenza e resistenza $R_r$, è possibile calcolare il valore della capacità e valutarne la compatibilità con la misura del multimetro. I dati raccolti sono presentati in tabella: 

\begin{table}[H]
\centering
\begin{subtable}{.5\textwidth}

\begin{tabular}{c|c}
\toprule
\multicolumn{2}{c}{Valori per il bilanciamento del ponte}\\
\midrule
\rowcolor{black!20}$R_r$ & $430 \pm 4.3  \ohm$ \\
$f$ & $ 1178.24\pm ?? Hz $ \\
\rowcolor{black!20}$C_x$ & $1.5682 \cdot 10^{-7} \pm 3.6349 \cdot 10^{-9} F $ \\
\bottomrule
\end{tabular}
\end{subtable}
\caption{gli errori sui valori sono stati calcolati a partire dall'errore dello strumento e per propagazione}
\end{table}

(compatibilità con il valore di C calcolato dal multimetro)

\subsection{Sensibilità del ponte a piccole variazioni di capacità}
Dall'analisi del circuito con il metodo simbolico risulta che il valore della differenza di tensione in uscita è pari a: 
\begin{equation}
V_{out}= V_{in} \biggl(\frac{1}{1+ \frac{R_r}{R_x} + 2 j\pi f R_r C_x } - \frac{1+j 2 \pi f R_2 C_2}{1 + j 2 \pi f (R_1 + R_2 ) C_2} \biggr)
\end{equation}
Questa espressione è molto utile ai fini del nostro studio perché siamo interessati a capire come varia V in uscita in funzione della capacità. Per questo motivo, esprimiamo $V_{out}$ in funzione della derivata parziale in $C_x$ e $R_x$: 
\begin{equation}
V_{out}= \frac{\partial {V_{out}}}{\partial {C_x}} \Delta{C_x} + \frac{\partial {V_{out}}}{\partial {R_x}} \Delta{R_x} = V_c + V_r
\end{equation}
Inoltre, facendo il calcolo delle derivate parziali a partire dall'equazione 4, risulta chiaro che $V_{out}$ sia ancora un'onda sinusoidale con la stessa frequenza ma sfasata. 
\begin{equation}
V_{out} = V_c \cdot sin(\omega t + \phi) + V_r \cdot cos(\omega t + \phi)
\end{equation}
Guardando queste equazioni risulta conveniente cercare di esprimere $\Delta{C_x}$ in funzione di $V_c$ e della derivata parziale, quindi tratteremo i dati nel seguente modo. Innanzitutto, cerchiamo il miglior fit dei nostri dati che segua l'equazione di un'onda sinusoidale: 
\begin{equation}
V_{out}=V_0 + V_{sin} (sin (\omega t)) + V_{cos} (cos(\omega t))
\end{equation}
Avendo a disposizione 10 serie di dati per ogni variazione di capacità, possiamo esprimere il valore finale dei coefficienti $V_{sin}$ e $V_{cos}$ con il valor medio e il suo errore con la deviazione standard. \newline \break
Il passo successivo consiste nella regressione lineare dei parametri appena trovati. Infatti, riprendendo l'equazione 6 e svolgendo l'argomento all'interno del seno e del coseno, si può riscrivere come:
\begin{equation}
V_{out}= sin(\omega t) (V_c cos \phi - V_r sin \phi ) + cos (\omega t) (V_r cos \phi + V_c sin \phi)
\end{equation}
Se si confronta l'espressione con l'equazione 7 risulta evidente che i termini $V_{sin}$ e $V_{cos}$ dipendono linearmente da $\Delta{C_x}$ secondo la legge:

$$V_{sin}= A_s + B_s \Delta_{C_x}$$
$$V_{cos} = A_c + B_c \Delta_{C_x}$$

dove 
$$ B_s = \frac{\partial{V_{out}}}{\partial{}C_x} cos \phi $$
$$ B_c =\frac{\partial{V_{out}}}{\partial{}C_x} sin \phi $$
A partire da questi coefficienti si può quindi trovare il valore della fase e della derivata in questo modo: 
\begin{equation}
\phi = \arctan{\frac{B_c}{B_s}}
\end{equation}
\begin{equation}
\frac{\partial{V_{out}}}{\partial{C_x}} = \sqrt{B_s^2 + B_c^2}
\end{equation}
A questo punto si possono utilizzare i valori ottenuti nell'equazione 6 e ottenere i valori sperimentali di $V_c$ e $V_s$ per regressione, e ottenere $\Delta{C_x}$ come:
\begin{equation}
\Delta{C_{mis}} = \frac{V_c}{\frac{\partial{V_{out}}}{\partial{C_x}}}
\end{equation}




\end{document}
